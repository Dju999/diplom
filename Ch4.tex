 \chapter{Содержание выполненной работы}

\section{Теоретическая часть работы}

В рамках теоретической части работы были проанализированы различные подходы, к исследованию времени, которое обучающийся тратит на ответ в системах компьютерного тестирования. Ряд подходов к данной проблеме  представлен в Главе \ref{ch2}. 

По результатам проведённого анализа в теоретической части работы для решения задачи, сформулированной в Введении к основной части дипломной работы, была выбрана одна из составляющих двухуровневой иерархической модели Ван дер Линдена - логнор\-мальная модель времени ответа первого уровня.

\section{Практическая часть работы}

Практическая часть выполненной работы состояла из двух основных час\-тей - это доработка существующей системы дистанционного обучения для решения задачи, постав\-ленной в дипломной работе и получения экспери\-ментальных данных, и обработка получен\-ных экспериментальных данных с целью адап\-тации работы системы дистанционного обуче\-ния на основании полученных экспериментальных данных.

\subsection{Доработка системы дистанционного обучения}

В процессе работы над поставленной задачей возникла необходимость доработки сис\-темы дистанционного обучения. На момент написания дип\-ломной работы система дистан\-ционного обучения не проводила учет времени, которое студент затрачивает при ответе на задачи в процессе работы с сис\-темой. Однако эта информация была нужна для проверки теоретических положений на практике. Для решения данной задачи были предприняты следующие шаги:

\begin{itemize}
\item построена диаграмма классов модуля системы дистанционного обучения
\item проанализаированы связи на диаграмме
\item спроектированы изменения, которые нужно внести в системы дистан\-ционного обуче\-ния для решения задачи
\item внесены изменения в программный код системы дистанционного обучения
\end{itemize}

В результате произведённых действий в системе дистанционного обучения появилась возможность фиксировать время, которое студент затрачивает для ответа на задачи в процессе обучения.

\subsection{Обработка экспериментальных данных}

При обработке экспериментальных данных были решены следующие задачи

\begin{itemize}
\item разработан алгоритм обращений к базе данных на языке запросов SQL
\item проведена первичная обработка экспериментальных данных
\item согласно разработанным алгоритмам была проведена оценка параметров модели
\end{itemize}

Для визуального анализа полученных статистических данных были пос\-троены гисто\-граммы.