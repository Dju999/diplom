\section{Содержание выполненной работы}

Выполненная дипломная работа состоит из двух основных разделов: тео\-ретической части работы и практической части работы.

В рамках теоретической части работы были проанализированы различные подходы, к исследованию времени, которое обучающийся тратит на ответ в системах компьютерного тестирования. Ряд подходов к данной проблеме  представлен в разделе \ref{ch22}. 

По результатам проведённого анализа в теоретической части работы для решения задачи, сформулированной в Введении, была выбрана одна из сос\-тавляющих двухуровневой иерархической модели ван дер Линдена - логнор\-мальная модель времени ответа.

Практическая часть выполненной работы состояла из двух основных час\-тей - это доработка существующей системы дистанционного обучения для решения задачи, постав\-ленной в дипломной работе, и получения эксперимен\-тальных данных для дальнейшей статистической обработки с целью оценки параметров модели. С использованием полученной модели был описан алго\-ритм выявления скомпроментированных задач.

Практическая часть работы включала в себя доработку системы дистан\-ционного обучения и обработку экспериментальных данных. В процессе ра\-боты над поставленной задачей возникла необходимость доработки сис\-темы дистанционного обучения. На момент написания дип\-ломной работы система дистан\-ционного обучения не производила учет вре\-мени, которое студент за\-трачивает при ответе на задачи в процессе работы с сис\-темой. Однако эта информация была нужна для проверки теоретических положений на практике. Для решения данной задачи были предприняты следующие шаги:

\begin{itemize}
\item построена диаграмма классов модуля системы дистанционного обучения
\item проанализаированы связи на диаграмме
\item спроектированы изменения, которые нужно внести в системы дистан\-ционного обуче\-ния для решения задачи
\item внесены изменения в программный код системы дистанционного обу\-чения
\end{itemize}

В результате произведённых действий в системе дистанционного обучения появилась возможность фиксировать время, которое студент затрачивает для ответа на задачи. Реализация указанных изменений была произведена с ис\-пользованиея языка веб-программирования PHP (который является основным языком разработки СДО МАИ CLASS.NET) и модификации базы данных MySQL, которую использует система.

При обработке экспериментальных данных были решены следующие за\-дачи:

\begin{itemize}
\item разработан алгоритм обращений к базе данных на языке запросов SQL
\item проведена первичная обработка экспериментальных данных
\item была проверена гипотеза о логнормальном распределении времени от\-вета
\item согласно разработанным алгоритмам была проведена оценка парамет\-ров модели
\end{itemize}

Для визуального анализа полученных статистических данных были пос\-троены гисто\-граммы. При статистической обработке данных использовался язык программирования Python и библиотека статистических методов stats.py.

\section{Задача конструирования ограниченных по времени тестов}

В качестве одного из приложений построенной модели можно привести задачу формирования индивидуальных заданий для ограниченных по вре\-мени тестов. Для корректной оценки знаний учащихся необходимо проведение ру\-бежного контроля, который включает в себя задачи разных типов. Обычно время проведения рубежного контроля ограничено - поэтому важно подобрать задачи таким образом, чтобы студенту хватило времени для его прохожения. Вероятностная модель времени ответа позволяет конструировать задания, которые учитывают индивидуальные особенности студентов.

Описаннная проблема сводится к следующей задаче линейного програм\-мирования с вероятностным ограничением (задача решается для конкретного студента, т.е. примем $t_{ij} \sim t_i$ для кратковсти):

$$
\left \{
\begin{array}{ll}
c-\sum\limits_{i=1}^{n}w_ix_i \rightarrow min & \\
\sum\limits_{i=1}^{n} x_i = k & \mbox{// количество задач в индивидуальном задании}\\
c - \sum\limits_{i=1}^{n} x_i \ge - \varepsilon & \mbox{// допустимая погрешность}\\
c - \sum\limits_{i=1}^{n} x_i \le \varepsilon & \mbox{// допустимая погрешность}\\
P\left( \sum\limits_{i=1}^{n} x_i t_{i} \le T \right) \ge \alpha & \mbox{// вероятностное ограничение на время теста}
\end{array}
\right.
$$

где

$$
\begin{array}{lcl}
c & - & \mbox{ требуемая сложность задания}\\
k & - & \mbox{ число задач в задании}\\
n & - & \mbox{ общее количество задач в пуле}\\
w_i & - & \mbox{ сложность задания i }\\
x_i & - & \mbox{ принадлежность задачи i тесту}\\
i & - & \mbox{ номер задания }, i=1,\ldots,K\\
j & - & \mbox{ номер студента, для которого генерируется задание}, j=1,\ldots,M\\
K & - & \mbox{ количество задач в пуле}\\
M & - & \mbox{ количество пользователей}
\end{array}
$$

Переменная $x_i$ является бинарным признаком принадлежности задачи тесту в следующем смысле:
$$
x_i = 
\left\{
\begin{array}{ll}
1, \mbox{ задача i попала в индивидуальное задание}\\
0, \mbox{ задача i не попала в индивидуальное задание}
\end{array}
\right.
$$

Так как случайные величины $t_{i}$, которые входят в вероятностное ограни\-чение,  распределены нормально и закон их распределения известен, то для данной задачи в стохастической по\-становке можно выписать детерминиро\-ванный эквивалент\cite{38.}:

$$
\left \{
\begin{array}{l}
c-\sum\limits_{i=1}^{n}w_ix_i \rightarrow min \\
\sum\limits_{i=1}^{n} x_i = k \\
c - \sum\limits_{i=1}^{n} x_i \ge - \varepsilon \\
c - \sum\limits_{i=1}^{n} x_i \le \varepsilon \\
\sum\limits_{i=1}^{n} x_{i}m_{i}  + \Phi^{-1}(1-\alpha)\sqrt{\sum\limits_{i=1}^{n} x_{i}^{2}\sigma^{2}_{i}} \le T 
\end{array}
\right.
$$
где $\Phi(x)$ - функция Лапласа. Т.к. время ответа имеет логнормальное распре\-деление, то и линейная комбинация (которая входит в вероятностное ограниче\-ние) так же распре\-делена нормально.