\section*{Реферат}

Дипломная работа содержит 67 страниц, 7 рисунков, 17 таблиц. Список использованных источников содержит 38 позиций.

{
\vspace{0.5cm}
АДАПТИВНОЕ КОМПЬЮТЕРНОЕ ТЕСТИРОВАНИЕ, ДИСТАНЦИ\-ОННОЕ ОБУЧЕНИЕ, МОДЕЛИРОВАНИЕ ВРЕМЕНИ ОТВЕТА, СИСТЕ\-МА ДИСТАНЦИОННОГО ОБУЧЕНИЯ, КОМПРОМЕНТАЦИЯ ЗАДАЧ, ОГРАНИЧЕННЫЕ ПО ВРЕМЕНИ ТЕСТЫ.
\vspace{0.5cm}
}

Дипломная работа посвящена проблеме адаптации системы дистанционного обучения на основании информации о времени ответа пользователя на задачи. Используется логнормальная модель врмени ответа.

Введение раскрывает актуальность, определяет степень разра\-ботки темы, объект, предмет,  цель и задачи исследования, раскрывает теоре\-тическую и прак\-тическую значимость работы.

В Главе 1 рассматриваются теоретические положения и аспекты предмета исследования, а так же их практичеcкое применение. Описаны исторические предпосылки создания стохастических моделей времени ответа. Предлагается математическая модель и аналитические оценки параметров модели. Произ\-водится проверка соответствия теоретической модели и реальных данных СДО МАИ CLASS.NET. Приводится решение двух прак\-тических задач: вы\-явление случаев <<нечестного>> прохождения  теста (когда студент пользуется заранее подготовленными ответами на одну или несколько задач теста) и задача формирования индивидуального задания требуемой сложности в ус\-ловиях ограниченного по времени теста.

Глава 2 дипломной работы содержит раздел <<Безопасность труда и охрана окружающей среды>>. В данной главе проис\-ходит анализ параметров поме\-щения, предназначенного для работы с сис\-темой ди\-станционнного обучения (компьютерный класс). Описаны негативные факторы, которые могут влиять на работу студентов, а так же указаны пути их устранения. 

Глава 3 предаставляет собой экономичес\-кий раздел. Произведён расчёт себестоимости реализации дипломной работы. Вычислена инвести\-ционная эффективность проекта и приведён срок его окупаемости для заказчика.

В Заключе\-нии формулируются итоги проведённого исследования и выводы по результатам выполненной дипломной работы.
