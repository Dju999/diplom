\chapter{Заключение} 

Задачи, сформулированные в разделе <<Введение>> выполнены: построена математическая модель времени ответа студента, на основании модели была решена задача прогнозирования времени ответа студента. По результатам прогноза описан алгоритм выявления откло\-нений при ответе на задачи с целью определить случаи, когда  студент имеет готовые ответы на одну или несколько задач теста.

Математическая модель получила программную реализацию и была при\-менена данным, полученным в системе дистанционного обучения МАИ.

Прогнозирование времени ответа студента является важной частью адап\-тивного компьютерного тестирования. Полученные результаты (математиче\-ская модель и программная реализация) могут быть применены не только к выявлению случаев мошенничества при прохождении тестирования, но и в ряде других задач: прогнозирование времени, которое понадобится студенту для прохождения теста; оценка вероятности выполнения теста в срок; про\-верка того, как формулировка задания влияет на скорость выполнения и выбор оптимальной формулировки.