{
\centering
\chapter{ОСНОВНАЯ ЧАСТЬ}
}
\setcounter{page}{4}
\newpage

\section{Введение в основную часть}
\label{intr}
\subsection{Актуальность дипломной работы}

В настоящее время одним из наиболее перспективных направлений раз\-вития в области образования является создание систем и методов дистанци\-онного обучения. Системы дистанционного обучения (СДО) позволяют ре\-шить проблему доступного образования: дистанционная модель (без присут\-ствия в образовательном учреждении) обеспечивает доступ к знаниям неза\-висимо места жительства и физических возможностей обучающегося. Сер\-висы дистанционного образования предоставляют как традиционные инсти\-туты (в России МГУ\cite{24.}, МИЛ\cite{25.}, МЭСИ\cite{26.}, за рубежом - MIT, Harvard, Stanford\cite{28.}), так и полностью виртуальные площадки в сети Интернет (Coursera\cite{27.}, Universarium\cite{29.} и т.д.).

В связи с развитием сервисов дистанционного обучения возникает необ\-ходимость в разработке математического и программного обеспечения для этих сервисов. Задачи, которые в возникают в данной области, а так же методы и пути их решения, получили название <<адаптивное компьютерное тестирование>> (Computer Adaptive Testing, CAT). Основной вопрос, на ко\-торый отвечают алгоритмы адаптивного компьютерного тестирования - как из пула задач подобрать задачу (или набор задач), которая максимально будет подходить для конкретного студента, т.е. СДО должна учитывать ин\-дивидуальные способности каждого обучающегося\cite{9.}. Такой подход к форми\-рованию индивидуального задания позволяет улучшить восприятие мате\-риала и способствует построению более эффективного процесса обучения.

Задача анализа времени ответа студента при ответе на задачи теста явля\-ется одним из приоритетных направлений области дистанционного обу\-чения и адаптивных систем тестирования. 

Время, которое студент затрачивает для ответа на задачу, является ос\-новным источником информации при ответе на следующие вопросы:
\begin{itemize}
\item не обладает ли студент ответами на некоторые (или все) задачи дистан\-ционного теста
\item достаточно ли отпущено времени для прохождения теста
\item удачно ли сформулированы задачи теста (или понимание условия за\-дачи вызывает затруднения)
\end{itemize}

На эти и некоторые другие вопросы помогает ответить стохастическая модель времени ответа в системах дистанционного обучения. Описанию такой модели и прикладных задач, которые решаются с её применением, посвящена данная дипломная работа.

{\itshape Объектом исследования} дипломной работы является поведение обуча\-ющихся при ответах на задания в системах дистанционного обучения. 

{\itshape Предметом исследования} дипломной работы является время, в течение которого студент отвечает на задачи, предлагаемые системой дистанционного обучения. 

\subsection{Цели и задачи}

Целью дипломной работы является адаптация системы дистанционного обучения на осно\-вании информации о времени, которое обучающийся затра\-чивает для ответа на задания теста

Основные задачи, которые необходимо решить для для достижения пос\-тавленной цели:
\begin{itemize}
\item ознакомление с теоретическим аспектом объекта исследования: поиск и чтение специализированной литературы
\item построение математической модели, описывающей предмет исследова\-ния
\item выбор методов оценки параметров построенных моделей
\item получение экспериментальных  данных
\item обработка данных моделирования эксперимента для оценки параметров математической модели
\item разработка алгоритма поиска отклонений в  новых данных, поступа\-ющих в систему дистанционного обучения, с помощью построенной мате\-матической модели для
\item описание и формализация задачи конструирования ограниченных по времени тестов заданной сложности.
\end{itemize}

Сформулированные задачи отражают процесс достижения поставленной цели дипломной работы.
%\subsubsection*{Элементы научной новизны в работе}

\subsection{Практическая значимость}

Дипломная работа имеет важное практическое приложение в качестве математического обеспечения для СДО МАИ CLASS.NET. Разработанные в работе методы и алгоритмы будут применяться для обеспечения учебного процесса пользователей данной системы.

В настоящее время созданию и развитию математического аппарата для СДО CLASS.NET посвящены работы \cite{14.,15.,16.}. Дипломная работа развивает идеи, предложенные в этих статьях, а так же предлагаются пути решения и формулировки для новых прикладных задач. 
