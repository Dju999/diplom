\chapter{Введение}

\label{intr}
\section{Актуальность дипломной работы}
Задача анализа времени ответа студента в системах дистанционного обу\-чения (СДО) является одним из приоритетных направлений области дистан\-ционного обучения и адаптивных систем тестирования. 
Время, которое студент затрачивает для ответа на задачу, является основным источником информации при ответе на следующие вопросы:
\begin{itemize}
\item не обладает ли студент ответами на некоторые (или все) задачи дистанци\-онного теста
\item достаточно ли отпущено времени для прохождения теста
\item удачно ли сформулированы задачи теста (или понимание условия задачи вызывает затруднения)
\end{itemize}

На все эти вопросы помогает ответить стохастическая модель времени ответа \\обучающегося в системах компьютерного обучения.

\section{Объект и предмет исследования}
{\itshape Объектом исследования} дипломной работы является поведение обуча\-ющихся при ответах на задания в системах дистанционного обучения. 

{\itshape Предметом исследования} дипломной работы является время, в течение которого студент отвечает на задачи, предлагаемые системой дистанционного обучения. 

\section{Цели и задачи}

Целью дипломной работы является оценка поведения студента на основании информа\-ции о времени, которое студент затрачивает для ответа на задания теста и выявлении отклонений во времени ответов студента обучающегося.

Основные задачи, которые решить для для достижения поставленной цели:
\begin{itemize}
\item ознакомление с теоретическим аспектом объекта исследования: поиск и чтение \\специализированной литературы
\item построение математических моделей, описывающих предмет исследо\-вания
\item выбор методов оценки параметров построенных моделей
\item получение данных экспериментальной выборки
\item обработка данных моделирования эксперимента для оценки параметров \\математической модели
\item решение поставленной цели - поиск отклонений в  новых данных, поступа\-ющих в систему дистанционного обучения, с помощью построенной мате\-матической модели
\end{itemize}

%\subsubsection*{Элементы научной новизны в работе}

\section{Практическая значимость}

Дипломная работа имеет важное практическое приложение в качестве математического обеспечения для cистемы дистанционного обучения МАИ.

\section{Краткое описание структуры}

Введение раскрывает актуальность, определяет степень научной разра\-ботки темы, объект, предмет,  цель, задачи и методы исследования, раскрывает теоре\-тическую и прак\-тическую значимость работы.
В Главе 1 рассматриваются теоретические положения и аспекты исследования времени, которое студент затрачивает для ответа на задачи в процес\-се обучения.
Глава 2 посвящена обработке экспериментальных данных и их последую\-щему анализу .
В Заключе\-нии формулируются итоги проведённого исследования и выводы по рассматри\-ваемой теме.