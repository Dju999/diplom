\documentclass[a4paper,14pt]{extreport} %размер бумаги устанавливаем А4, шрифт 12пунктов
\usepackage[T2A]{fontenc}
\usepackage[utf8]{inputenc}%включаем свою кодировку: koi8-r или utf8 в UNIX, cp1251 в Windows
\usepackage[english,russian]{babel}%используем русский и английский языки с переносами
\usepackage{amssymb,amsfonts,amsmath,cite,enumerate,float} %подключаем нужные пакеты расширений

%для таблиц длиннее, чем на страницу
\usepackage{longtable}

%для большой таблицы, на несколько страниц
\newenvironment{myTable}%
{%
    \tiny
    \begin{longtable}[H]{|p{0.9cm}|p{3.4cm}|p{1.2cm}|p{3.9cm}|p{0.6cm}|p{0.7cm}|p{0.6cm}|p{0.7cm}|}
    \hline
}%
{%
    \end{longtable}
}%



\begin{document}
\begin{myTable}
\hline
Этап & Событие             & Шифр & Работа                                        & $t_{min}$ & $t_{max}$ & $t_{exp}$ & $\sigma$\\
\hline
1    & Изучение литературы & 1-4  & Изучение литературы по основной части диплома & 2         & 5         &  3,8      &  0,36    \\
\cline{3-8}
&

&
1-12&
Изучение литературы по экономической части хитровыебанногодипломаска\-сукасукасукасукаорсиоывсоывисоывор&
3&
6&
4,8&
0,36
\\
\cline{3-8}
&

&
1-10&
Изучение литературы по охране труда и окружающей среды&
1&
4&
2,8&
0,36
\\
\hline
2&
Окончание доработки СДО&
2-7&
Анализ структуры и программного кода СДО&
6&
9&
7,8&
0,36
\\
\hline
3&
Получен список необходимых доработок&
3-2&
Доработка СДО для получения экспериментальных данных и оценки параметров модели&
4&
8&
6,4&
0,64
\\
\hline
4&
Необходимость построения математической модели&
4-5&
Изучение математических методов моделирования времени ответа в СДО&
2&
5&
3,8&
0,36
\\
\hline
5&
Завершение математической модели&
5-6&
Применение математической модели для анализа времени ответа&
7&
11&
9,4&
0,64
\\
\cline{3-8}
&

&
5-2&
Описание математической модели&
6&
9&
7,8&
0,36
\\
\hline
7&
Оценка параметров модели&
7-8&
Оценка по экспериментальным данным параметров модели &
7&
10&
8,8&
0,36
\\
\cline{3-8}
&

&
7-9&
Применение модели к реальным входным данным&
2&
5&
3,8&
0,36
\\
\hline
8&
Прогнозирование ответа на основании модели&
8-9&
Получение результатов обработки реальных данных&
3&
6&
4,8&
0,36
\\
\hline
9&
Построение таблиц, диаграмм, анализ полученных данных&
9-14&
Оформление результатов практической части&
4&
7&
5,8&
0,36
\\
\hline
10&
Раздел "Охрана труда и окружающей среды"&
10-11&
Написание теоретического материала раздела "Охрана труда и окружающей среды"&
3&
5&
4,2&
0,16
\\
\hline
11&
Расчет кондиционирования&
11-14&
Проведение расчётов&
6&
8&
7,2&
0,16
\\
\hline
12&
Раздел "Экономическая эффективность"&
12-13&
Написание теоретического материала раздела "Экономическая эффективность"&
2&
5&
3,8&
0,36
\\
\hline
13&
Построение сетевого графика&
13-14&
Проведение расчётов&
6&
9&
7,8&
0,36
\\
\hline
14&
Оформление готового диплома&
-
&
Оформление полученных результатов&
&

&

&
\\
\hline
\end{myTable}

\end{document} 
