 \chapter{Охрана труда и окружающей среды}
\label{mainpart}  
\section{Введение}

\subsection{Необходимость защиты труда в объекте дипломной работы}

Дипломная работа посвящена анализу времени, которое затрачивает обу\-чающийся при ответе на задачи в системе дистанционного обучения. Одно из достоинств электронной версии методических и учебных материалов состоит в том, что студент может проходить обучение как в домашней обстановке, так и на территории института (в компьютерном классе). 

Как известно, на скорость студента при ответе влияет не только способ\-ность самого студента к обучению, но и факторы помещения, в котором проходит процесс обучения, - отсюда следует, что при проведении занятий на территории института необходимо  обеспечить в помещениях, где проходит работа с системой дистанционного обучения, условия для продуктивной ра\-боты. Необходимо учесть вредные факторы, которые могут возникнуть у студентов в процессе работы и устранить эти факторы, а так же причину их возникновения

\subsection{Характеристики рабочего помещения}
\label{hro}

Рабочее помещение, в котором проводятся занятия, имеет следующие ха\-рактеристики:

\begin{itemize}
\item {\itshape длина,} м : 10
\item {\itshape ширина,} м : 15
\item {\itshape площадь,} м$^2$ : 150
\item {\itshape высота,} м : 5
\item {\itshape рабочих мест,} шт. : 20
\item {\itshape площадь на одного студента,} м$^2$. : 7,5
\item {\itshape объём на одного студента,} м$^3$. : 37,5
\end{itemize}

\subsection{Характеристики оборудования}

Помещение, в котором проводятся занятия, оснащено персональными ком\-пьютерами в следующей комплектации:
\begin{itemize}
\item системный блок (основные комплектующие, производящие шумовое воз\-действие – вентилятор и жесткий диск)
\item клавиатура
\item мышь
\item соединительные провода
\item монитор
\end{itemize}

Каждое рабочее место оснащается одним персональным компьютером.

В таблице приведён уровень шума и мощность для каждого комплек\-тующего

\begin{table}[H]
\begin{center}
\begin{tabular}{|l|c|c|}
\hline
Комплектующее &Уровень шума(дБ) & Мощность оборудования(Вт)\\
\hline
Жесткий диск & 35 & 10\\
\hline
Вентилятор & 40 & 15\\
\hline
Клавиатура & 0 &  0.75\\
\hline
Мышь & 0 & 1 \\
\hline
Соединительные провода & 0 & 0.1 \\
\hline
Монитор & 15 & 25\\
\hline
\end{tabular}
\end{center}
\end{table}

\section{Анализ условий труда}

\subsection{Санитарно-гигиенические факторы}
\label{sangen}

\subsubsection{Микроклимат}

Микроклимат - искусственно создаваемые условия микросреды в зак\-рытых помещениях.
Микроклимат необходимо поддерживать, для создания комфорт\-ных условий работы - влажности, скорости движения воздуха, тем\-пературы, давления и т.д.

Для обеспечения нормального микроклимата применяются:

\begin{itemize}
\item кондиционирование
\item вентилирование
\item обогрев
\end{itemize}

Требования к микроклимату определяет ГОСТом 30494-96 «Здания жи\-лые и общест\-венные. Параметры микроклимата в помещениях»  и ГОСТом 12.01.005-88 «Общие санитарно-гигиенические требования к воздуху рабочей зоны». Учебные аудито\-рии согласно ГОСТу относятся к помещениям 2-ой категории/

Согласно классификации, приведённой в ГОСТ 12.01.005-88, работа с сис\-темой дистан\-ционного обучения, относится к категории работ 1а – это лёгкие работы, т.е. работы, производимые сидя и сопровождающиеся незначитель\-ным физическим напряжением.

В таблице приведены нормативные значения и фактические значения в помещении, где проходит работа с сис\-темой дистанционного обучения:

\begin{table}[H]
\begin{center}
\begin{tabular}{|l|p{1,3cm}|p{2.7cm}|p{3.2cm}|p{3cm}|}
\hline
Период года &  & Температура воздуха $^\circ$С & Относительная влажность, \% & Скорость движения воздуха, м/с(не более)\\
\hline
Холодный & ГОСТ & 22-24 & 40-60 & 0.1\\
%\cline{2-5}
 & Факт & 26 & 55 & 0.05\\
\hline
Тёплый & ГОСТ & 23-25 & 40-60 & 0.1\\
%\cline{2-5}
 & Факт & 27 & 60 & 0.08\\
\hline
\end{tabular}
\end{center}
\end{table}

Таким образом, фактические условия микроклимата в помещении, где используется система дистанционного обучения, не соответствуют требова\-ниям ГОСТ. 12.01.005-88, поэтому необходимо рассчитать параметры кондици\-онирования для устранения негативных факторов.

\subsubsection{Освещение}

Правильно спроектированное и рационально исполненное освещение помо\-гает создать комфортные психофизиологические условия для длительной работы - поэтому освещён\-ность является одним из важнейших факторов для комфортного протекания про\-цесса обучения. Для улучшения видимости объектов освещение должно быть рав\-номерным. Так же в поле зрения сту\-дента не должно быть резких переходов от света к тени. Большое количество глянце\-вых поверхностей по возможности нужно заменить матовыми, чтобы избежать блёсткости.

Нормы освещения в образовательных помещениях определяются СанПиН\\ 2.2.1/2.1.1.1278-03 «Гигиенические требования к естественному, искусствен\-ному и сов\-мещенному освещению жилых и общественных зданий». В по\-мещении используется совме\-щенный комбинированный тип освещения. Срав\-ним фактические показатели со стан\-дартом:

\begin{table}[H]
\begin{center}
\begin{tabular}{|p{3cm}|p{3cm}|p{3cm}|}
\hline
& \multicolumn{2}{|c|}{Освещённость (лк)}  \\
\cline{2-3}
& всего & в т.ч. от общего \\
\hline
СанПиН & 500 & 300\\
\hline
Факт & 510 & 305\\
\hline
\end{tabular}
\end{center}
\end{table}

Показатели помещения соответствуют значениям СаНПиН 2.2.1/2.1.1.1278-03 и в корректировке не нуждаются.

\subsubsection{Электроопасность}

Действие электрического тока на организм может быть разнообразным: ожоги, механи\-ческие повреждения кожи, изменение биологических процессов организма.

Электротравмы  подразделяются на общие и местные. Общая травма – это электри\-ческий удар. Приводит к судорогам, остановке дыхания, нарушению сердечной деятель\-ности. К местным травмам относят ожоги (термические повреждения), металлизацию кожного покрова, механические повреждения тканей (разрыв тканей в результате  электро\-динамического эффекта).

Для гигиенического нормирования электроопасности оборудования ис\-пользуется ГОСТ 12.1.038-82, «Электробезопасность. Предельно допустимые значе\-ния напряжений  прикос\-новения и токов», который устанавливает пре\-дельно допустимые токи, протекающие через тело человека во время при\-косновения к электроустановкам.
При работе с системой дистанционного обу\-чения в сис\-темных блоках протекает переменный ток частотой 50 Гц. Для такого уровня тока при действии в течении 0.6 сек. ГОСТ 12.1.038-82 опреде\-ляет предельно допустимый уровень напряжения в 125 В. При этом напря\-жение на корпусе системного блока составляет 100 В – таким образом, требо\-вания ГОСТ соблюдены.

\subsubsection{Шум}

Звук - это акустические колебания упругой среды. {\itshape Акустическими} на\-зывают колебания, которые может воспринять человек с нормальным слухом, т.е. колебания в диапазоне частот $16$ Гц - $209$ кГц. Звуковые колебания, распро\-страняющиеся в пространстве, предста\-вляют собой акустическое поле.

{\itshape Шумом} называют совокупность акустических звуков различной интен\-сивности и часто\-ты (см. \cite{10.}). Интенсивный шум на производстве является негативным фактором: шум способстует увеличению числа ошибок в процессе работы, так как приводит к снижению внимания. Так же шум негативно влияет на скорость принятия решений, затрудняет протекание аналитических процессов.

В биологическом отношении шум представляет собой стрессовый фактор, который может вызвать срыв припособительных способностей организма, т.н. акустический стресс. Акустический стресс может приводить к разнообразным расстройствам, от расстройств центральной нервной системы до морфоло\-гический деструктивных изменений в органах и тканях. Степень негативного влияния шума зависит от продолжительности воздействия, уровня интенсив\-ности шума, функционального состояния нервной системы человека и индиви\-дуальной чувствительности конкретного человека в данному виду раздражи\-теля (что очень важно - например, женский и детский оргаизм более чувстви\-тельны к шуму). Высокая индивидуальная чувствительность к шуму может становиться причиной развития неврозов, других расстройств нервной сис\-темы, а так же быстрой утомляемости.

Шум оказывает сильное влияние различные аспекты функционирования организма человека: угнетение ЦНС, нарушение дыхания, сбои пульса, может стать причиной нарушения обмена веществ, возникновению сердечно-сосу\-дистых заболеваний, развитию различных профессиональных болезней.

Уровень звукового давления принято измерять в децибелах (дБ). Децибелл - отно\-сительная единица измерения звукового давления. Измерение уровня звука проиcходит по отношению к опорному давлению $p_0=20$ мкПа, которое соответствует порогу слы\-шимости синусоидальной звуковой волны частотой $1$ кГц. Уровень звукового давления в децибеллах $N$ для давления $p$ вычисля\-ется по формуле
$$
N = 20 \log \frac{p}{p_0}
$$

Особо выделяют следующие уровни шума:
\begin{itemize}
\item {\itshape 30 \ldots 35} дБ : привычен для человека и не беспокоит его;
\item {\itshape 40 \ldots 70} дБ : возникает значительная нагрузка на нервную систему, ухудшение самочувствия;
\item {\itshape 70 \ldots 140} дБ : может привести к потере слуха;
\item {\itshape 140 \ldots 160} дБ : разрыв барабанных перепонок и контузия;
\item свыше {\itshape  160} дБ : смерть.
\end{itemize}

Норму параметров шума на рабочих местах определяет ГОСТ 12.1.003-831 «Шум. Общие требования безопасности» с дополнениями 1989 г. Документ классифицирует шумы по спектру (широкополосные и тональные) и по вре\-менным характеристикам (постоянные и непостоянные). Для нормирования постоянных шумов используются допустимые уровни звукового давления \\(УЗДН) в девяти активных полосах частот, разделённых по видам произ\-водственной деятельности. В свою очередь непостоянные шумы делятся на три груп\-пы:

\begin{itemize}
\item колеблющиеся
\item прерывистые
\item импульсные
\end{itemize}

Сравним фактические данные с нормами ГОСТ 12.1.003-83 на рабочих местах, связа\-нных с обучением. Для этого произведём расчёт фактического уровня шума согласно таблице пункта \ref{hro}

Произведём расчёт звукового давления оборудования по каждому источ\-нику:
$$
\mbox{Жесткий диск: } p_1= 10^{35/20}\cdot 2 \cdot 10^{-4} = 0.01
$$
$$
\mbox{Вентилятор: } p_2= 10^{40/20}\cdot 2 \cdot 10^{-4} = 0.02
$$
$$
\mbox{Монитор: } p_3= 10^{15/20}\cdot 2 \cdot 10^{-4} = 0.001
$$

С учётом проведённых расчётов, вычислим результирующий уровень шу\-ма:
$$
N = 20 \log \frac{\sum\limits_{n}^{i=1}p_i}{p_0} =  20 \log \left(\frac{0.01 + 0.02 + 0.001}{2 \cdot 10^{-4}}\right) = 
$$
$$
=  20 \log \frac{0.031}{2 \cdot 10^{-4}} = 44 \mbox{ дБ}
$$

Проверим полученный результат на соответствие ГОСТ

\begin{table}[H]
\begin{center}
\begin{tabular}{|p{4cm}|p{3cm}|}
\hline
Деятельность: обучение & Уровень шума (дБ)  \\
\hline
ГОСТ & 50  \\
\hline
Факт & 44  \\
\hline
\end{tabular}
\end{center}
\end{table}

Таким образом, фактический уровень шума соответствует нормативному.

\subsubsection{Вибрация}

Вибрации - это механические колебания, которые возникают и распро\-страняются в упругих средах. Воздействие вибрации на человека разделяют по способу передачи коле\-баний (общая и локальная), по направлению дей\-ствия (по оси $x$, по оси $y$, по оси $z$) и по временным характеристикам (по\-стоянная, непостоянная).

Между частотой воздействующей вибрации и реакциями организма нет линейной зави\-симости. Причина - в эффекте резонанса. Если внешняя частота вибраций совпадает с собственными вибрация тела, воздействие становится крайе негативным.

Особенно сильное воздействие вибрация оказывает на зрение. Расстрой\-ство зрительного восприятия происходит при частотном диапазоне от $60$ до $90$ Гц. Патологии, связанные с вибрацией по количеству связанных с ними профес\-сиональных заболеваний стоят на втором месте после пылевых патологий.

Так же при действии вибрации на организм страдает нервная система, вестибюлярный и тактильный анализаторы. У людей, длительное время под\-вергающихся вибрации, отме\-чают головокружения, снижение остроты зрения.

Гигиеническое нормирование вибрации производится на основании ГОСТ 12.1.012-90 <<Вибрационная безопасность. Общие требовани>> (воспользуемся эквивалентными скор\-ректированными значениями).

\begin{table}[H]
\begin{center}
\begin{tabular}{|p{4.1cm}|p{2cm}|p{1cm}|p{2cm}|p{1cm}|}
\hline
Среднегеометрические частоты полос, Гц & \multicolumn{4}{|c|}{Допустимые значения по осям $X_0, Y_0, Z_0$  } \\
\cline{2-5}
       & \multicolumn{2}{|l|}{виброускорения                   }&\multicolumn{2}{|l|}{ виброскорости                    } \\
\cline{2-5}
       & $10^{-3} \mbox{ м}^2 / \mbox{с}$  & дБ & $10^{-3} \mbox{ м}^2 / \mbox{с}$  & дБ   \\
\hline
СанПиН & 10                                & 80 & 0.28                              & 75   \\
\hline
Факт   & 9.8                               & 77 & 0.25                              & 70   \\
\hline
\end{tabular}
\end{center}
\end{table}

Таким образом, значения вибрации соответствуют нормам ГОСТ.

\subsubsection{Электромагнитные излучения}

В зависимости от энергии фотонов выделяют ионизирующие и неиони\-зирующие излу\-чения. Среди неионизирующих в свою очередь выделяют элек\-трические и магнитные поля.

Нормирование ЭМИ промышленной частоты производится путём уста\-новления допус\-тмых уровней напряжённости электрического поля $E$(кВ/м) и напряженности маг\-нитного поля $H$(А/м).

Пребывание в электрическом поле напряжённостью до $5$ кВ/м допускается в течении всего рабочего дня. Допустимое время (в часах) пребывания в ЭП напряжённостью $5 \ldots 10$ кВ/м вычисляется по формуле
$$
T = \frac{50}{E} - 2,
$$
где $E$ - напряжённость воздействующего поля в контролируемой зоне, кВ/м.

Нормирование значение напряженности определяется СанПиН 2.2.4.1191-03 <<Электро\-магнитные поля в производственных условиях>>. Т.к. металличес\-кий корпус системного блока экранирует магнитные поля и прошел соответ\-ствующую сертификацию на стороне производителя, то нормы СанПиН вы\-полняются.

\subsection{Эргономика рабочего места}

\label{erg}

Требования к эргономике рабочего места закреплены в санитарно-эпиде\-миологических нормах СанПиН 2.2.2/2.4.1340-03 <<НАЗВАНИЕ>>. Рабочее место студента должно отвечать требованиям, обеспечивающим достаточную степень эрго\-номичности. Для обеспечения этих требований должны выполняться опре\-делённые условия: 
\begin{itemize}
\item оптимальное раcположение оборудования;
\item достаточный объём рабочего пространства.
\end{itemize}

Рабочее место студента, оснащённое персональным компьютером, имеет следующие параметры эргономики: 
\begin{itemize}
\item высота рабочей поверхности должна регулироваться в пределах $680-850$ мм, при отсутствии регулировки — 725мм
\item размеры пространства для ног должны не должны вызывать диском\-форт
\item расположение документов на рабочем месте
\item Кресло должно быть подъемно-поворотным, регулируемым по высоте, углам наклона сиденья и спинки
\item поверхность рабочего стола
\item возможность регулировки элементов рабочего места
\end{itemize}

Удобная рабочая поза студента помогает избежать перенапряжения мышц и спо\-собствует улучшению кровотока и дыхания. При неудобной рабочей позе могут появиться боли в мышцах, суставах и сухожилиях. Так же среди программистов, проводящих большое количесто времени на неэргономичном рабочем месте, развивается туннельный синдром запястья.
Правильное по\-ложение в рабочем кресле включает в себя:
\begin{itemize}
\item прямую посадку
\item опора - только на спинку кресла
\item расслабленную позу без излишнего напряжения в поясничном отделе
\item голова  - немного наклонена (до 20 градусов)
\item руки расслаблены, локти держать под углом 90 градусов, кисти рук — на уровне локтей или немного ниже
\item колени должны быть на уровне бедер, стопы - на подставке
\end{itemize}

Соблюдение норм эргономики, описанных выше, позволит снизить вред\-ное воздействие ЭВМ на пользователя персонального компьютера, повысить вни\-мательность, снизить наг\-рузки на зрительные и слуховые рецепторы.

\subsection{Психофизиологические факторы}

К психофизиологическим факторам относятся перегрузки (физические и эмоциональ\-ные), перенапряжение связанное с интеллектуальной работой и монотонностью труда. Психофи\-зиологические факторы могут вызывать перенапряжение как в физическом, так и в интел\-лектуальном плане.

Оценка психофизиологических факторов проводится с помощью руковод\-ства P2.2.2006-05 <<Руководство по гигиенической оценке факторов рабочей среды и трудового процесса. Критерии и классификация условий труда>>

В руководстве P2.2.2006-05 под пунктом 5.10. <<Тяжесть и напряженность трудового процесса>> с помощью Таблицы 18 <<Классы условий труда по по\-казателям» напряженности трудового процесса» определим класс физичес\-кого труда>>.

Проведём анализ класса условий труда с использованием соответству\-ющих таблиц руководства P2.2.2006-05:

\begin{longtable}[H]{|p{5.5cm}|p{6cm}|p{3.0cm}|}
\hline
Показатель & Значение & Класс условий труда \\
\hline
\multicolumn{3}{|l|} {1.Интеллектуальные нагрузки:}\\
\hline
1.1. Содержание работы & Решение простых задач по инструкции & Допустимый \\
\hline
1.2.Распределение функций по степени сложности задания & Обработка, выполнение задания и его проверка & Допустимый\\
\hline
1.3.Характер выполняемой работы & Работа по установленному графику с возможной его коррекцией по ходу деятельности & Допустимый\\
\hline
\multicolumn{3}{|l|} {2.Сенсорные нагрузки}\\
\hline
 2.1.Длительность сосредоточенного наблюдения(\%времени)&26-50&Допустимый\\
\hline
2.2.Наблюдение за экранами видеотерминалов (часов)& до З &Допустимый\\
\hline
\multicolumn{3}{|l|} {3.Эмоциональные нагрузки}\\
\hline
З.1.Степень ответственности за результат собственной деятельности. Значимость ошибки& Несет ответственность за функциональное качество выполения заданий. Влечет за собой дополнительные усилия со стороны преподавателя&Допустимый\\
\hline
3.4.Количество конфликтных ситуаций (за занятие)&1-3&Допустимый\\
\hline
\multicolumn{3}{|l|} {4. Монотонность нагрузок}\\
\hline
4.1. Число элементов (приемов), необходимых для реализации простого задания или в многократно повторяющихся операциях&более 10&Оптимальный\\
\hline
4.2. Продолжительность (в сек) выполнения простых заданий или повторяющихся операций& более 100&Оптимальный\\
\hline
4.3. Время активных действий (в \% к продолжительности смены). В остальное время – наблюдение&19 - 10&Допустимый\\
\hline
Монотонность производственной обстановки (время пассивного наблюдения), \% &76–80&Допустимый\\
\hline
\end{longtable}

Согласно таблицам, класс труда определяем как «Допустимый». Этот уровень соответ\-ствует напряжённости труда средней степени.

\section{Расчёт}

Согласно СНиП 41-01-2003 «Отопление, вентиляция и кондиционирование воздуха», расход воздуха на одного человека в помещениях, находящихся в общественных зданиях без естественного проветривания составляет 60 м$^3$/ч. Исходя из этой нормы, рассчитаем величину кондиционирования по формуле (СНиП 41-01-2003? Приложение М, формула И.1), используя величину из\-бытков явной теплоты:
$$
L = L_{w,z} + \frac{3.6Q - cL_{w,z}(t_{w,z} - t_{in})}{c(t_{l}-t_{in})},
$$
где

$L_{w,z}$ - расход воздуха, удаляемого из обслуживаемой или рабочей зоны помещения системами местных отсосов, и на технологические нужды,  $\mbox{м}^3/\mbox{ч}$

$Q$ - избыточный явный тепловой поток в помещении, ассимилируемый воздухом центральных систем вентиляции и кондиционирования, Вт

$t_{w,z}$ - температура воздуха, удаляемого системами местных отсосов в об\-служиваемой или рабочей зоне помещения, и на технологические нужды, $^\circ$C

$t_{in}$ - температура воздуха, подаваемого в помещение $^\circ$C

$t_{l}$ - температура воздуха, удаляемого из помещения за пределами обслу\-живаемой или рабочей зоны, $^\circ$C

$c$ - теплоемкость воздуха, равная 1,006 $\mbox{кДж}/(\mbox{кгС})$

Произведём расчёт избыточного теплового потока Q. Избыточный тепло\-вой поток определяется по формуле
$$
Q = Q_{ob} + Q_{l} + Q_{os} + Q_{ok},
$$
где 

$Q_{ob}$ - теплота, выдёляемая оборудованием. В соответствии с характерис\-тиками обору\-дования из пункта 1.1.3 «Характеристики оборудования», по\-лучаем.
$$
Q_{ob} = P_{ob}\cdot f = (10+15+0.75+0.1+0.25)\cdot 0.25 = 13 \mbox{ Вт},
$$
где

$P_{ob}$ - номинальная мощность оборудования (Вт), $f=0.25$ - коэффициент передачи

 $Q_{l}$ -теплота, выделяемая людьми в помещении. При этом $Q_{l} = 50\cdot \frac{1000}{860} = 58 \mbox{Вт}$, т.к. один человек выделяет $50 \mbox{ ккал/час}$

$Q_{os}$ - теплота, выделяемая системой освещения
$$
Q_{os} = P_{os} \cdot a \cdot b \cdot \cos(f) = 20 \cdot 4 \cdot 0.46 \cdot 1 \cdot 0.3 = 11 \mbox{Вт},
$$
где

$a = 0.46$ - коэффициент перехода электрической энергии в световую

$b = 1$ -  коэффициент одновременной работы ламп

$\cos(f) = 0.3$ - коэффициент мощности

$P_{os} = 20 \mbox{ Вт}$ -  номинальная мощность освещения одной лампы освещения
(всего таких ламп 4 шт.)

$Q_{ok}$ - теплота, выделяемая ограждающими конструкциями
$$
Q_{ok} = F\cdot q_{ok} = 18 \cdot 11 \cdot \frac{1000}{860} = 230 \mbox{ Вт},
$$
где

$F = 18 \mbox{ м}^2$  - площадь ограждающей конструкции, излучающей тепло

$q_{ok} = 11 \mbox{ ккал/м}^2$ -  теплота, которую излучает 1 м$^2$ ограждающей кон\-струкции.

Таким образом, получаем избыточный тепловой поток
$$
Q = Q_{ob} + Q_{l} + Q_{os} + Q_{ok}= 13 + 58 + 11 + 230 = 312 \mbox{Вт}
$$

Рассчитаем требуемую величину кондиционирования:
{\small
$$
L = L_{w,z} + \frac{3.6Q - cL_{w,z}(t_{w,z} - t_{in})}{c(t_{l}-t_{in})} = 2 + \frac{3.6\cdot312\cdot3600 - 1006\cdot2\cdot(22-19)}{1006(24-19)} = 804
$$
}

Согласно произведённым расчётам и ГОСТ 26963—86 <<Кондиционеры бытовые авто\-номные. Общие технические условия>>, необходимо выбрать сле\-дующий кондиционер: тип КБ1 и климатическое исполнение У3, обозначение климатического исполнения по ГОСТ 15150-69 «Исполнение для различных климатических районов».

\section{Вывод}

Помещение, которое используется для обучения студентов с использо\-ванием системы СДО отвечает условиям безопасности труда - это подтвер\-ждается проведёнными расчётами. 

Согласно произведённым расчётам рекомендуется выбрать кондиционер КБ1 У3, тогда параметры микроклимата в помещении будут соответствовать принятым нормам.


 
