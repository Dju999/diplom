 \section{Результаты}
 \label{ch5}

В дипломной работе приведены предпосылки к необходимости модели\-рования времени, которое студент затрачивает при ответе на задачи в системе дистанционного обучения. Формулируется математическая вероятностная мо\-дель времени ответа, для которой приводятся аналитические оценки пара\-метров.

Корректность модели и степень её применимости проверяется на реальных данных СДО МАИ CLASS.NET. Для этого в программный код системы вносятся изменения для получения необходимой статистической информации о пользователях. Внесённые доработки позволя\-ют фиксировать время ответа студента и сохранять его для дальнейшей обработки. Добавлением новых таблиц изменена структура базы данных, а так же модифицирован алго\-ритм внесения информации о работе пользователя в БД. Разработана программная библиотека на языке программирования Python для доступа к базе данных и работы со статистической информацией. По результатам обработки экспе\-риментальных данных был сделан вывод о соот\-ветствии математической мо\-дели и реальных данных СДО МАИ CLASS.NET.

Согласно сформулированной математической модели была произведена оценка параметров по реальным данным с использованием разработанного программного модуля. Была разработана процедура прогнозирования вре\-мени ответа пользователя.

По результатам сравнивания прогнозных значений с фак\-тическими про\-изводится вывод о том, не обладал ли студент ответами к решаемой задаче. Описан алгоритм принятия решения, использующий понятие доверительного интервала.

На основании решения о компроментации задачи можно принимать раз\-личные административные решения: назначить пользователю решение до\-полнительных задач, убрать скомпроментированные задачи из пула для дан\-ной группы студентов и т.д.

В качестве одного из приложений описанной математической модели при\-ведена задача конструирования ограниченных по времени тестов с учётом индивидуальных способностей студента.