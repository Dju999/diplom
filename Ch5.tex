 \chapter{Результаты}
 \label{ch5}

Доработки, внёсенные в систему дистанционного обучения, позволяют фиксировать время, которое студент затрачивает при решении задач в системе дистанционного обучения.

По результатам обработки экспериментальных данных был сделан вывод о соответствии математической модели реальным данным.

Согласно сформулированной математической модели была произведена оценка параметров модели.

С использованием накопленной информации о времени ответа студентов на задания в системе дистанционного обучения была разработана процедура прогнозирования времени ответа студента.

По результатам прогноза (при сравнивании прогнозных значений с фак\-тическими) производится вывод о том, не обладал ли студент ответами к решаемой задаче.
