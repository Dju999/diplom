 \chapter{Экономический раздел}

 \section{Введение}

 Целью дипломной работы является доработка системы дистанционного обучения МАИ с целью адаптации системы на основе поступающей информа\-ции о времени, которое студент затрачивает для ответов на задачи в процессе тестирования. На основании анализа времени ответа делается вывод о том, не было ли у студента заранее приготовленных ответов не задачи теста.
В основной части диплома описывается математическое обеспечение системы и программная реализация, которые представляют собой завершенный прог\-раммный продукт, готовый для продажи на рынке программного обеспечения.
В экономической части диплома производится расчёт затрат, которые несёт разработчик программного обеспечения системы дистанционного обучения МАИ. Исходя из затрат формируется стоимость проекта в целом. По результа\-там оценки стоимости проекта производится расчёт экономической эффектив\-ности.

\section{Сетевой график}

Сетевой график представляет собой графическое отображение этапов работ, которые необходимо провести для завершения дипломного проекта. С помощью сетевого графика можно рассчитать спрогнозировать общее время, которое потребуется для выпол\-нения проекта, а так же вычислить вероятность выпол\-нения проекта в срок.

\subsection{Таблица этапов работ}
По каждому этапу дипломной работы указано минимальное время выпол\-нения этапа и максимальное время выпол\-нения этапа. Исходя из этих значений рассчитывается ожидаемое время ответа:
$$
t_{exp} = \frac{2t_{min}+3t_{max}}{5}
$$

Дисперсия времени работы вычисляется по формуле
$$
t_{exp} = \left( \frac{t_{max} - t_{min}}{5} \right)^2
$$

По результатам расчётов построим таблицу
\begin{myTable}
\hline
Этап & Событие             & Шифр & Работа                                        & $t_{min}$ & $t_{max}$ & $t_{exp}$ & $\sigma$\\
\hline
1    & Изучение литературы & 1-4  & Изучение литературы по основной части диплома & 2         & 5         &  3,8      &  0,36    \\
\cline{3-8}
&

&
1-12&
Изучение литературы по экономической части&
3&
6&
4,8&
0,36
\\
\cline{3-8}
&

&
1-10&
Изучение литературы по охране труда и окружающей среды&
1&
4&
2,8&
0,36
\\
\hline
2&
Окончание доработки СДО&
2-7&
Анализ структуры и программного кода СДО&
6&
9&
7,8&
0,36
\\
\hline
3&
Получен список необходимых доработок&
3-2&
Доработка СДО для получения экспериментальных данных и оценки параметров модели&
4&
8&
6,4&
0,64
\\
\hline
4&
Необходимость построения математической модели&
4-5&
Изучение математических методов моделирования времени ответа в СДО&
2&
5&
3,8&
0,36
\\
\hline
5&
Завершение математической модели&
5-6&
Применение математической модели для анализа времени ответа&
7&
11&
9,4&
0,64
\\
\cline{3-8}
&

&
5-2&
Описание математической модели&
6&
9&
7,8&
0,36
\\
\hline
7&
Оценка параметров модели&
7-8&
Оценка по экспериментальным данным параметров модели &
7&
10&
8,8&
0,36
\\
\cline{3-8}
&

&
7-9&
Применение модели к реальным входным данным&
2&
5&
3,8&
0,36
\\
\hline
8&
Прогнозирование ответа на основании модели&
8-9&
Получение результатов обработки реальных данных&
3&
6&
4,8&
0,36
\\
\hline
9&
Построение таблиц, диаграмм, анализ полученных данных&
9-14&
Оформление результатов практической части&
4&
7&
5,8&
0,36
\\
\hline
10&
Раздел "Охрана труда и окружающей среды"&
10-11&
Написание теоретического материала раздела "Охрана труда и окружающей среды"&
3&
5&
4,2&
0,16
\\
\hline
11&
Расчет кондиционирования&
11-14&
Проведение расчётов&
6&
8&
7,2&
0,16
\\
\hline
12&
Раздел "Экономическая эффективность"&
12-13&
Написание теоретического материала раздела "Экономическая эффективность"&
2&
5&
3,8&
0,36
\\
\hline
13&
Построение сетевого графика&
13-14&
Проведение расчётов&
6&
9&
7,8&
0,36
\\
\hline
14&
Оформление готового диплома&
-
&
Оформление полученных результатов&
&

&

&
\\
\hline
\end{myTable}

\subsection{Построение сетевой модели}

Построим на основании таблицы из п. 1.2.1 «Таблица этапов работ» сетевой график. Круги обозначают события, стрелками обозначены работы. Подписи над стрелками указывают ожидаемое время выполнения работы.
\\
ЗДЕСЬ ГРАФИК
\\
\subsection{Анализ сетевой модели.}

Путём в сетевой модели называется последовательность работ, соединя\-ющая две любые работы на графике (если такая последовательность сущест\-вует)
Введём обозначения: $L_{1(i)}$ - путь, предшествующий событию $i$, $L_{2(i)}$ - путь, следующий за событием $i$.

Критический путь на сетевой модели (последовательность событий от начала проекта к концу проекта, имеющая максимальную длину):
$$
L = 1-4-5-6-3-2-7-8-9-14
$$

Длина пути $T(L) = \sum\limits_{i-j}t_{i-j}$ (где $t_{i-j}$ - длительность работы с меткой $i-j$) - сумма длительностей работ, которые выполняются при следовании по пути. Длина критического пути:
$$
T_{\mbox{кр}}(L) = 3.8+3.8+9.4+6.4+6.4+7.8+8.8+4.8+5.8 = 50.6
$$
Ранний срок наступления события: $t_{\mbox{р}(i)} = max(T(L_{1(i)}))$\\
Ранний срок начала работы: $t_{\mbox{рн}(i-j)} = max(T(L_{1(i)}))$\\
Ранний срок окончания работы: $t_{\mbox{ро}(i-j)}= t_{\mbox{р}(i)} - t_{(i-j)} = max(T(L_{1(i)})) + t_{(i-j)} $\\
Поздний срок наступления события: $t_{\mbox{п}(i)}= T_{\mbox{кр}}(L) -  max(T(L_{2(i)}))$\\
Поздний срок начала работы: $t_{\mbox{пн}(i-j)}= t_{\mbox{по}(i)} - t_{(i-j)}$\\
Поздний срок окончания работы: $t_{\mbox{по}(i-j)} = t_{\mbox{п}(j)}= T_{\mbox{кр}}(L) -  max(T(L_{2(i)}))$\\
Общий резерв времени работы: $R_{(i-j)} = t_{\mbox{по}(i-j)} - t_{\mbox{ро}(i-j)} =  t_{\mbox{п}(j)} -  t_{\mbox{р}(j)} -  t_{(i-j)}$\\
Свободный резерв времени работы: $r_{(i-j)} = t_{\mbox{р}(i-j)} - t_{\mbox{ро}(i-j)} =  t_{\mbox{р}(j)} -  t_{\mbox{р}(i)} -  t_{(i-j)}$\\
Резерв времени события: : $r_{(i)} = t_{\mbox{п}(i)} - t_{\mbox{р}(i)} $

Результат расчёта параметров для сетевой модели работы над дипломным проектом представлен в таблице (утолщённым шрифтом обозначены работы, принадлежащие критическому пути):
\begin{myTableSecond}
\hline
Шифр & $t_{exp}$ & $\sigma$ &  $t_{\mbox{рн}(i-j)}$ $t_{\mbox{р}(i)}$ & $t_{\mbox{ро}(i-j)}$ & $t_{\mbox{пн}(i-j)}$ & $t_{\mbox{по}(i-j)}$ $t_{\mbox{п}(j)}$ & $R_{(i-j)}$ & $r_{(i-j)}$ & $r_{(i)}$\\
\hline
1-4&
3,8&
0,36&
0&
3,8&
8,8&
12,6&
8,8&
0&
8,81\\
\hline
1-12&
4,8&
0,36&
0&
4,8&
49,4&
54,2&
49,4&
0&
49,4\\
\hline
1-10&
2,8&
0,36&
0&
2,8&
51,6&
54,4&
51,6&
1,4&
51,6\\
\hline
2-7&
7,8&
0,36&
29,8&
37,6&
38,6&
46,4&
8,8&
0&
8,8\\
\hline
3-2&
6,4&
0,64&
23,4&
23,4&
32,2&
38,6&
15,2&
6,4&
15,2\\
\hline
4-5&
3,8&
0,36&
3,8&
7,6&
12,6&
16,4&
8,8&
2,7&
8,8\\
\hline
5-6&
9,4&
0,64&
7,6&
17&
16,4&
25,8&
8,8&
9,6&
8,8\\
\hline
5-2&
7,8&
0,36&
15,4&
15,4&
30,8&
38,6&
23,2&
14,4&
23,2\\
\hline
6-3&
6,4&
0,64&
17&
23,4&
25,8&
32,2&
8,8&
0&
8,8\\
\hline
6-7&
9,6&
0,04&
26,6&
26,6&
36,8&
46,4&
19,8&
11&
19,8\\
\hline
7-8&
8,8&
0,36&
37,6&
46,4&
46,4&
55,2&
8,8&
0&
8,8\\
\hline
7-9&
3,8&
0,36&
37,6&
51,2&
56,2&
60&
8,8&
0&
8,8\\
\hline
8-9&
4,8&
0,36&
46,4&
51,2&
55,2&
60&
8,8&
0&
8,8\\
\hline
9-14&
5,8&
0,36&
51,2&
57&
60&
65,8&
8,8&
0&
8,8\\
\hline
10-11&
4,2&
0,16&
2,8&
7&
54,4&
58,6&
51,6&
0,2&
51,6\\
\hline
11-14&
7,2&
0,16&
7&
14,2&
58,6&
65,8&
51,6&
0&
51,6\\
\hline
12-13&
3,8&
0,36&
4,8&
8,6&
54,2&
58&
49,4&
0&
49,4\\
\hline
13-14&
7,8&
0,36&
8,6&
16,4&
58&
65,8&
49,4&
0&
49,4\\
\hline
\end{myTableSecond}

Директивный срок выполнения проекта составляет 122 дня, при этом длина критического пути 50.6 дней – таким образом, проект будет завершён в строк и нет необходимости перестраивать сетевой график проекта. Сумма дисперсий работ, лежащих на критическом пути, составляет 4.08. Среднеквад\-ратическое отклонение для критического пути составляет $\sqrt{4.08} = 2.02$ . Доверительный интервал для срока выполнения всех работ имеет вид $[50.6-2.02,50.6+2.02 ] \sim [30.58, 52.62]$. Вероятность выполнения работы в срок составляет $P=\Phi((122-50.6)/2.02)=\Phi(35.3)\sim 1$, где $\Phi(x)$ – функция Лапласа.

\section{Расчет затрат}

В разделе описаны основные затраты разработчика, влияющие на цену конечного продукта: расходные материалы, аренда помещений, заработная плата и т.д.

\subsection{Приобретение материалов}

Для проведения процесса разработки программного обеспечения необхо\-димо приобрести оборудование для разработчика (ноутбук) и пакет прик\-ладных программ для разработчика.

\subsubsection{Оборудование}
Для выбора оборудования произведём сравнительный анализ нескольких моделей с использованием сервиса Яндекс.Маркет (http://market.yandex.ru/).
Сравнение произведём между тремя моделями стоимостью до 22 000 руб. от производителей “Lenovo”, “Dell” “Samsung”.

\begin{table}[H]
\begin{center}
\begin{tabular}{|p{3.5cm}|p{3.6cm}|p{3.6cm}|p{3.4cm}|}
\hline
Характеристика & \multicolumn{3}{|c|}{Модель}\\
\hline
Название&
Lenovo THINKPAD L420&
DELL Vostro 1440&
Samsung 535U4C\\
\hline
Операционная система&
Win 7 Professional&
Win 7 Home Basic 64&
Win 7 Home Basic 64\\
\hline
Тип процессора&
Core i3&
Celeron&
Core i3\\
\hline
Частота процессора (МГц)&
2300&
2000&
1600\\
\hline
Размер оперативной памяти&
2&
2&
4\\
\hline
Тип экрана&
матовый&
матовый&
Глянец\\
\hline
Объём накопителя&
250&
320&
500\\
\hline
Время работы&
11&
8&
7\\
\hline
Графика&
интегрированная&
интегрированная&
дискретная\\
\hline
Вес&
2.24&
2.19&
1.81\\
\hline
Цена&
12 952&
13 855&
21 459\\
\hline
\end{tabular}
\end{center}
\end{table}

Т.к. работать с ноутбуком планируется не в одном и том же месте, а при постоянных перемещениях, то модель от Lenovo не подходит, т.к. обладает слишком большим весом. Так же Lenovo имеет самый маленький размер жесткого диска среди представленных моделей, а объём оперативной пяти у него не больше, чем у конкурентов.

Среди двух оставшихся моделей Samsung обладает более мощным гра\-фическим процессором (дискретным), большим объёмом жёсткого диска и малым весом, а так же большим объемом оперативной памяти. При этом Dell имеет более производительным процессором Intel, более долгим временем работы от аккумулятора, а так же матовым экраном (это более эргономично для программиста), а так же меньшей ценой. Поэтому принимаем решение о покупке ноутбука DELL Vostro 1440 стоимостью N = 21459 руб. на основании сравнительного анализа.

\subsubsection{Программное обеспечение}

Цикл разработки программного обеспечения включает в себя несколько этапов: анализ требований, проектирование системы, разработка програм\-много обеспечения. В таблице приведены затраты на минимально необходимый список программного обеспечения по каждому этапу. Цены приведены в рублях, по курсу на 24.11.2013

\begin{table}[H]
\begin{center}
\begin{tabular}{|p{3.5cm}|p{3.6cm}|p{3.6cm}|p{3.4cm}|}
\hline
Название этапа&
Название ПО&
Цена ПО (руб.)\\
\hline
Анализ требований&
Trello&
1320\\
\hline
Проектирование системы&
Microsoft Visio&
19 499\\
\hline
Разработка ПО&
Sublime Text 3&
2310\\
\hline
\multicolumn{2}{|c|}{Итого} &23219\\
\hline
\end{tabular}
\end{center}
\end{table}

Итого, на приобретение программного обеспечения будет затрачено U = 23219 руб.

\subsection{Аренда помещений}

По причине того, что разработкой системы дистанционной системы обу\-чения занимается программист, который тратит на разработку не полный рабочий день, в качестве помещения для разработки будет использован ковор\-кинг-цетр недалеко от метро Шаболовская (т.к. это наиболее удобный для программиста район). Ссылка на сайт центра: http://www.matrixoffice.ru. Ком\-ната в коворкинг-центре включает всё, что нужно для работы: кресло, стол, высокоскоростной доступ в интернет.
В таблице произведён расчёт затрат на аренду помещения с учётом того, что цикл разработки займёт три месяца (сентябрь, октябрь, ноябрь)

\begin{table}[H]
\begin{center}
\begin{tabular}{|p{3.5cm}|p{3.6cm}|p{3.6cm}|}
\hline
Число рабочих дней на проект&
63\\
\hline
Количество рабочих часов в день&
4\\
\hline
Полное число часов&
252\\
\hline
Стоимость часа аренды&
200\\
\hline
Итого&
50400\\
\hline
\end{tabular}
\end{center}
\end{table}

Итого, затраты на аренду составят H = 50400 руб.

\subsection{Заработная плата}

При работе над экономической частью дипломного проекта необходимо рассчитать заработную плату сотрудникам, задействованным в работе над дипломом.

В работе над дипломным проектом принимают участие научный руко\-водитель, программист, консультанты. Для каждого специалиста вычисляется объём заработной платы, исходя из почасовой ставки и  времени, в течение которого специалист участвует  проекте

Расчёты приведены в таблице: Данные для вычисления почасовой зара\-ботной платы программиста получены с помощью сайта http://www.hh.ru/, данные по заработной плате научных руководителей и консультантов соот\-ветствуют зарплатам в Московском авиационном институте за 2013 г.
\begin{table}[H]
\begin{center}
\begin{tabular}{|p{3.5cm}|p{1.8cm}|p{1.6cm}|p{2.8cm}|p{2.1cm}|p{1.8cm}|}
\hline
Специалист&
Заработ\-ная плата&
Коли\-чество часов &
Количество часов на одного дипломника&
Почасовая оплата&
Заработ\-ная плата\\
\hline
Научный руководитель&
40000&
90&
24&
444&
10666\\
\hline
Консультант по экономической части&
40000&
90&
2&
444&
888\\
\hline
Консультант по охране труда и окружающей среды&
40000&
48&
2&
833&
1666\\
\hline
Программист&
-&
252&
-&
630&
158760\\
\hline
\multicolumn{5}{|c|}{Итого}&
171980\\
\hline
\end{tabular}
\end{center}
\end{table}

Итого затраты на заработную плату составят Z = 171980 руб.

\subsection{Транспортные расходы}

Затраты на транспорт фигурируют в работе над дипломным проектом, т.к. встречи между участниками проекта происходят на территории МАИ и каждый из работников затрачивает денежные средства на то, чтобы добраться до МАИ. Предполагается, что все участники добираются до МАИ на метро.

В таблице для каждого участника учтено количество поездок (определяется исходя из этапов работы над дипломом), а так же полные затраты на транспорт за период написания диплома:
\begin{table}[H]
\begin{center}
\begin{tabular}{|p{3.5cm}|p{2.8cm}|p{2.6cm}|p{2.8cm}|}
\hline
Специалист&
Количество поездок&
Стоимость 1-ой поездки&
Итого\\
\hline
Научный руководитель&
16&
&
448\\
\cline{1-2}\cline{4-4}
Консультант по экономической части&
2&
&

56
\\
\cline{1-2}\cline{4-4}
Консультант по охране труда и окружающей среды&
2&
28
&
56
\\
\cline{1-2}\cline{4-4}
Программист&
20&
&
560\\
\hline
\multicolumn{3}{|c|}{Итого}&
1120\\
\hline
\end{tabular}
\end{center}
\end{table}

Итого расходы на транспорт составляют M = 1120 руб.

\subsection{Расходы амортизацию оборудования}

Оборудование, которое использовалось в ходе выполнения дипломной ра\-боты, подлежит амортизации. Амортизации подвергаются основные средства и нематериальные активы для переноса части их стоимости в цену производи\-мой продукции.

Согласно пункту 1.2.1.1 «Оборудование», основным оборудованием (сред\-ством производства) является ноутбук. Расчет амортизации произведём ли\-нейным способом:
$$
A = \frac{S\cdot n \cdot t}{ 100 \cdot T}
$$

В таблице приведены расшифровка и значения переменных в формуле.
\begin{table}[H]
\begin{center}
\begin{tabular}{|p{3.0cm}|p{3.8cm}|p{2.6cm}|}
\hline
Переменная&
Описание&
Значение\\
\hline
S&
Стоимость оборудования&
21459 руб.\\
\hline
n&
Годовая норма амортизации&
20 \%\\
\hline
t&
Время работы оборудования&
252 часа\\
\hline
T&
Эффективный срок работы оборудования&
1800 часов\\
\hline
\end{tabular}
\end{center}
\end{table}

Произведём расчёт согласно данным таблицы:
$$
A = \frac{21459\cdot 20 \cdot 252}{ 100 \cdot 1800} = 601 \mbox{ (руб.)}
$$

Таким образом, амортизация ноутбука за период написания диплома со\-ставит 601 руб.

\section{Социальные отчисления}

Работодателю необходимо произвести социальные отчисления с зарплаты работников в следующем размере:
Взносы в ФФОМС: 5.1\%\\
Взносы в ФСС: 2.9\%\\
Взносы в Пенсионный фонд: 22\%

Таким образом, на затраты на социальные отчисления составят (с учётом фонда оплаты труда из пункта 1.2.3 «Заработная плата»
$$
P = Z \cdot (5.1 + 2.9 + 22) = Z \cdot 30 = 171980 \cdot 30 = 51594 \mbox{ (руб.)}
$$

Итого, размер социальных выплат составит 51594 руб.

\subsection{Прочие расходы}

К прочим расходам относится мобильная связь. Исходя из того, что один комплект оператора «МТС», включающий 300 минут разговоров и неограни\-ченное количество смс стоит 500 руб. для каждого участника разработки диплома, то за 4 месяца для 4-х человек получаем
$$
B = 500 \cdot 4 \cdot 4 = 8000 \mbox{ (руб.)}
$$

Итого, общие расходы составляют 8000 руб.

\subsection{Накладные расходы}

К накладным расходам относятся затраты, не связанные прямо с разработ\-кой системы дистанционного обучения – например, приобретение литературы для разработчика. Накладные расходы принимаются в размере 5\% от фонда оплаты труда:
$$
D = FOT \cdot 5\% = 171980 \cdot 5 \% = 8599 \mbox{ (руб.)}
$$

Итого, накладные расходы составляют $8599 \mbox{ (руб.)}$

\section{Расчет экономической эффективности}

Для расчёта экономической эффективности нужно вычислить се\-бесто\-имость продукта, его цену продажи, а так же экономический эффект, который ожида\-ется от внедрения продукта

\subsection{Расчёт цены на продукт}

Чтобы рассчитать цену на продукт, необходимо вычис\-лить его себесто\-имость. Се\-бестоимость продукта с определяется как сумма всех затрат, вычис\-ленных в пункте 1.2 «Расчет затрат»:\\
\begin{math}
SS = D + B + P + A+M+H+U+N+Z = \\ 
=171980 + 8000 + 601 + 1120 + 50400 + 23219 + 21459 + 8599 + 51594 = 336972
\end{math}

Норма прибыли равна 5\%. НДС, который так же нужно учесть в цене, составляет 18\%. Тогда цена на продукт составит
$$
C = 336972 \cdot (100\% + 5\% + 18\%) = 414476 \mbox{ (руб.)}
$$

\section{Экономический эффект}

Экономический эффект, который принесёт внедрение доработок в систему дистанционного обучения, заключается в ликвидации недополученной прибыли за дополнительные занятия для студентов.

После внедрения системы появиться возможность выделить среди группы студентов, проходящих курсы в системе дистанционного обучения, тех обу\-чающихся, которые решают задачи теста с заранее имеющимися ответами.

Если студент использует  готовые ответы – он не до конца освоил нужный курс и нуждается в дополнительных занятиях. Оценим прибыль, которую институт может получить за один семестр по формуле:
$$
L = N \cdot i \cdot k \cdot V
$$

В таблице приведены расшифровка и значения переменных в формуле.
В таблице приведены расшифровка и значения переменных в формуле.
\begin{table}[H]
\begin{center}
\begin{tabular}{|p{3.0cm}|p{4.1cm}|p{2.6cm}|}
\hline
Переменная&
Описание&
Значение\\
\hline
L&
Количество студентов &
430\\
\hline
i&
Доля воспользовавшихся ответами&
15 \%\\
\hline
k&
Стоимость одного доп. занятия на курсах&
540 руб.\\
\hline
V&
Количество доп. занятий&
4 шт
\\
\hline
\end{tabular}
\end{center}
\end{table}

Таким образом, получаем:

Итого, экономический эффект от системы дистанционного обучения сос\-тавит 139320 руб. в семестр.

\subsection{Экономическая эффективность}

Для расчёта экономической эффективности необходимо учесть расходы на внедрение системы и эксплуатацию сервиса.

Расходы на внедрение системы  вычисляются по формуле 
$$
C_{\mbox{вн}} = t_{\mbox{вн}}\cdot p,
$$
где $t_{\mbox{вн}} = 3$ часа - время, необходимое для установки системы на сервер и $p=630$ руб. - почасовая ставка программиста, который занимается развёр\-тыванием системы
$$
C_{\mbox{вн}} = 3 \cdot 630 = 1890 \mbox{ (руб.)},
$$

Расходы на эксплуатацию сервиса складываются из затрат на оплату сервера, который использует система дистанционного обучения:
$$
C_{\mbox{эксп}} = t_{\mbox{эксп}}\cdot c,
$$

где $t_{\mbox{эксп}} = 6$ мес. (расчёты производим за семестр) и  $c = 1200$ руб. – помесячная оплата сервера.
$$
C_{\mbox{эксп}} = 6 \cdot 1200 = 7200 \mbox{ (руб.)},
$$ 

Итого получаем срок окупаемости вложенных средств:
$$
T_{\mbox{ок}} = \frac{C_{\mbox{эксп}} + C_{\mbox{вн}} + C}{L} = \frac{7200+1890+414476}{139320} = 3.04 (\mbox{ семестра})
$$

Таким образом средства, вложенные в систему, будут возвращены в течение четырёх семестров.

\section{Вывод} 
В экономической части дипломной работы был произведён расчёт себес\-тоимости и цены продажи проекта.

Построен сетевой график выполнения дипломной работы. По графику найден критический путь и рассчитаны параметры сетевой модели для каждо\-го узла. По рассчитанным параметрам произведена оценка наиболее вероят\-ного срока выполнения проекта, а так же вычислена вероятность завершения проекта в срок.

Была проведена оценка затрат на введение и эксплуатацию системы, а так же оценка прибыли, которую принесёт проект. На основании этих данных определёна экономическая эффективность разработанной системы.

По результатам расчётов сделан вывод о том, что вложенные в систему средства будут возвращены инвестору в течении четырёх учебных семестров.