%меняем "Литература" на корректное название
\renewcommand{\bibname}{Список использованных источников}
\addcontentsline{toc}{chapter}{Список использованных источников}

\begin{thebibliography}{2}
\bibitem{24.}  {\itshape Интернет-портал дистанционного образования МГУ: [Электронный ресурс] // Управление развития, перспективных проектов и непрерывного образования МГУ}, 2014 г. Режим доступа свободный: http://www.distance.msu.ru/
\bibitem{25.}  {\itshape Центр дистанционного обучения Московского института лингвистики: [Электронный ресурс] // Московский институт лингвистики}, 2014 г. Режим доступа свободный: http://www.inyaz-mil.ru/courses/formi-obucheniya/tsentr-distantsionnogo-obucheniya-mil
\bibitem{26.}  {\itshape Центр онлайн-обучения в МЭСИ: [Электронный ресурс] // Московский государственный университет экономики, статистики и информатики}, 2014 г. Режим доступа свободный: http://mesi.ru/education/higher/zaochnoe-on-lin/
\bibitem{28.}  {\itshape Сборник онлайн курсов edX: [Электронный ресурс] // edX}, 2014 г. Режим доступа свободный: https://www.edx.org/
\bibitem{27.}  {\itshape Образовательная организация Coursera: [Электронный ресурс] // Coursera. 2014 г}, Режим доступа свободный: https://www.coursera.org/
\bibitem{29.}  {\itshape Межвузовская площадка электронного образования Универсариум: [Электронный ресурс] // Универсариум}, 2014 г. Режим доступа свободный: http://universarium.org/
\bibitem{9.} Wim J. van der Linden, {\itshape Using Response Times for Item Selection in Adaptive Testing // Journal of Educational and Behavioral Statistics}, 2008, Vol. 33. No. 7, pp. 5-20
\bibitem{14.} Наумов А.В., Иноземцев А.О., {\itshape Алгоритм формирования индивидуальных заданий в системах дистанционного обучения. // Вестник компьютерных и информационных технологий, 2013, №6, сc. 46-51}
\bibitem{15.} Иноземцев А.О., Кибзун А.И., {\itshape  Оценивание уровней сложности тестов на основе метода максимального правдоподобия. // Автоматика и телемеханика, 2014, №4 (принята к публикации) }.
\bibitem{16.} Кибзун А.И. Панарин С.И., {\itshape Формирование интегрального рейтинга с помощью статистической обработки результатов тестов. // Автоматика и Телемеханика, 2012, № 6, сс. 119-139.}
\bibitem{30.} Woodbury M. A., {\itshape On the standard length of a test // Psychometrika}, 1951, № 16, сс. 103–106
\bibitem{31.} Woodbury M. A., {\itshape The stochastic model of mental test theory and an application}, 1963, № 28, сс. 391–393
\bibitem{32.} Lord, F. M., Novick, M. R., {\itshape Statistical theories of mental test scores // Addison Wesley}, 1968
\bibitem{33.} Gulliksen H., {\itshape Theory of mental tests. New York: Wiley}, 1950
\bibitem{34.} Thurstone L. L. {\itshape Ability, motivation, and speed. Psychometrika}, 1937, № 2, cc. 249–254
\bibitem{18.} Steven L. Wise and Lingling Ma {\itshape Setting Response Time Thresholds for a CAT Item Pool: The Normative Threshold Method // National Council on Measurement in Education, Vancouver, Canada}
\bibitem{17.} Gaviria, J.-L., {\itshape Increase in precision when estimating parameters in computer assisted testing using response times // Quality \& Quantity}, 2005, № 39, cc. 45–69
\bibitem{19.} Wise, S. L., Kong, X., {\itshape Response time effort: A new measure of examinee motivation in computer-based tests// Applied Measurement in Education}, 2005 , № 18, сс. 163-183
\bibitem{20.} Wang T., Hanson B.A. , {\itshape Development and Calibration of an Item Response Model that Incorporates Response Time// American Educational Research Association} 2001
\bibitem{8.} Wim J. van der Linden, {\itshape   Conceptual Issues in Response-Time Modeling} 
\bibitem{35.} Roskam, E. E., {\itshape Toward a psychometric theory of intelligence // Progress in mathematical psychology },1987,  сс. 151–171
\bibitem{36.} Rasch G.,  {\itshape Probabilistic models for some intelligence and attainment tests // Chicago: University of Chicago Press}, 1960
\bibitem{1.} Wim J. van der Linden, {\itshape Some New Developments in Adaptive Testing Technology // Journal of Psychology}, 2008, Vol. 216(1), pp. 3–11
\bibitem{7.} Wim J. van der Linden, {\itshape   Predictive Control of Speededness in Adaptive Testing // Law School Admission Council Computerized Testing Report}, 2007
\bibitem{22.} Rasch G. {\itshape Probabilistic models for some intelligence and attainment tests // The University of Chicago Press}, 1980.
\bibitem{21.} Thissen D., {\itshape Timed testing: An approach using item response theory // New York: Academic Press.} 1983
\bibitem{2.} Rob R. Meijer, Leonardo S. Sotaridona, {\itshape Detection of Advance Item Knowledge Using Response Times in Computer Adaptive Testing // Law School Admission Council Computerized Testing Report}, 2006
\bibitem{6.} Wim J. van der Linden, David J. Scrams, Deborah L. Schnipke, {\itshape  Using Response-Time Constraints to Control for Differential Speededness in Computerized Adaptive Testing // Applied Psychological Measurement}, 1999, Vol. 23 No. 3, pp. 195–210
\bibitem{23.} Кибзун А.И., Наумов А.В., Горяинова Е.Р., {\itshape  Теория вероятностей и математическая статистика. Базовый курс с примерами и задачами // Под ред. Кибзуна А.И. - М.: ФИЗМАТЛИТ}, 2007, 234 стр.
\bibitem{3.} Wim J. van der Linden, {\itshape   A Lognormal Model for Response Times on Test Items // Journal of Educational and Behavioral Statistics}, 2006, Vol. 31, No. 2, pp. 181–204
\bibitem{5.} Wim J. van der Linden, Edith M.L.A. Van Krimpen-Stoop, {\itshape  Using Response Times to Detect Aberrant Responses in Computerized Adaptive Testing // Psychometrika}, 2003, Vol. 68, No. 2, pp. 251-265
\bibitem{38.} Юдин Д.Б., {\itshape Математические методы управления в условиях неполной информации // Советское радио}, 1974, 400 c
\bibitem{11.} D'Agostino R.B., Pearson E. S., {\itshape Testing for departures from normality // Biometrika}, 1973, No 60, pp. 613-622
\bibitem{12.} D'Agostino R.B., {\itshape An omnibus test of normality for moderate and large sample size // Biometrika}, 1971, No 58, pp. 341-348
\bibitem{13.} Кобзарь А.И., {\itshape Прикладная математическая статистика}
\bibitem{37.} {\itshape информационно-аналитический ресурс, посвященный машинному обучению: [Электронный ресурс] // MachineLearning.ru}, 2014 г. Режим доступа свободный: http://www.machinelearning.ru/wiki/index.php
\bibitem{4.} Wim J. van der Linden, {\itshape  Modeling response times with latent variables: Principles and applications// Psychological Test and Assessment Modeling}, 2011, Vol. 53, pp. 334-358
\bibitem{10.} С.В. Белов, В.А. Девисилов, А.В, Ильницкая и др., {\itshape   Безопасность жизнедеятельность // М.: Высш. шк.}, 2007

\end{thebibliography}